% !TeX spellcheck = pl_PL
\documentclass[a4paper]{article}
\usepackage{polski}
\usepackage[utf8]{inputenc}
\usepackage{graphicx}
\usepackage{amsmath}

\usepackage[unicode, bookmarks=true]{hyperref} %do zakładek
\usepackage{tabto} % do tabulacji
\NumTabs{6} % globalne ustawienie wielkosci tabulacji
\usepackage{array}
\usepackage{multirow}
\usepackage{array}
\usepackage{dcolumn}
\usepackage{bigstrut}
\usepackage{color}
\usepackage[usenames,dvipsnames]{xcolor}
\usepackage{wrapfig}
\usepackage{listings,lstautogobble}
\usepackage[usenames,dvipsnames]{xcolor}
\usepackage{titlesec}

\makeatletter
\@addtoreset{section}{part}
\makeatother

\setlength{\textheight}{24cm}
\setlength{\textwidth}{15.92cm}
\setlength{\footskip}{10mm}
\setlength{\oddsidemargin}{0mm}
\setlength{\evensidemargin}{0mm}
\setlength{\topmargin}{0mm}
\setlength{\headsep}{5mm}
\newcommand{\HRule}{\rule{\linewidth}{0.5mm}}
\usepackage{endnotes}
\def\enoteheading{\par\kern2\baselineskip%
	\footnoterule%
	\kern1\baselineskip}
\let\footnote\endnote
\def\footnotetext{\endnotetext[\number\numexpr\value{endnote}+1]}
\let\footnotemark\endnotemark 

% --- Opcje listingu kodu
\lstset{
	frame=single,
	autogobble=true,
	commentstyle=\ttfamily\itshape\color{ForestGreen},
	keywordstyle=\color{blue},
	frameround=ffff,
	rulecolor=\color{black},
	tabsize=4,
	breaklines=true, %
	% --- Polskie znaki w listingu kodu
	literate=%
	{ą}{{\c{a}}}1
	{ć}{{\'c}}1
	{ę}{{\c{e}}}1
	{ł}{{\l{}}}1
	{ń}{{\'n}}1
	{ó}{{\'o}}1
	{ś}{{\'s}}1
	{ź}{{\'z}}1
	{ż}{{\.z}}1
	{Ą}{{\c{A}}}1
	{Ć}{{\'C}}1
	{Ę}{{\c{E}}}1
	{Ł}{{\L{}}}1
	{Ń}{{\'N}}1
	{Ó}{{\'O}}1
	{Ś}{{\'S}}1
	{Ź}{{\'Z}}1
	{Ż}{{\.Z}}1
}
% -- listing SiSzarpa
\definecolor{bluekeywords}{rgb}{0,0,1}
\definecolor{greencomments}{rgb}{0,0.5,0}
\definecolor{redstrings}{rgb}{0.64,0.08,0.08}
\definecolor{xmlcomments}{rgb}{0.5,0.5,0.5}
\definecolor{types}{rgb}{0.17,0.57,0.68}
\lstset{
	language=[Sharp]C,
	captionpos=b,
	%numbers=left, %Nummerierung
	%numberstyle=\tiny, % kleine Zeilennummern
	frame=lines, % Oberhalb und unterhalb des Listings ist eine Linie
	showspaces=false,
	showtabs=false,
	breaklines=true,
	showstringspaces=false,
	breakatwhitespace=true,
	escapeinside={(*@}{@*)},
	commentstyle=\color{greencomments},
	morekeywords={partial, var, value, get, set, Double, DateTime},
	keywordstyle=\color{bluekeywords},
	stringstyle=\color{redstrings},
	basicstyle=\ttfamily\small,
}

\usepackage{titlesec}

\begin{document}
\bibliographystyle{plain}

	\begin{titlepage}
		\begin{center}
			
			% Upper part of the page. The '~' is needed because \\
			% only works if a paragraph has started.
			\includegraphics[width=0.5\textwidth]{./img/logo.png}~\\[1cm]
			%?[width=0.15\textwidth]
			
			\textsc{\LARGE Politechnika Śląska w Gliwicach}\\[1.5cm]
			
			\textsc{\Large Biologically Inspired Artificial Intelligence}\\[0.5cm]
			
			% Title
			\HRule \\[0.4cm]
			{ \huge \bfseries Przewidywanie kursu walutowego  \\[0.4cm] }
			
			\HRule \\[1.5cm]
			
			% Author and supervisor
			\textsc{\Large Autorzy:} \\
			Forczmański Mateusz \\
			Szukała Patryk\\[1.0cm]
			
			Informatyka, semestr VI \\
			Rok akademicki 2014/2015 \\
			Grupa GKiO3
			
			\vfill
			
			% Bottom of the page
			{\large \today}
			
		\end{center}
	\end{titlepage}
	{\hypersetup{hidelinks}
		\tableofcontents
	}
	\newpage
	
	\part{Temat projektu}
		\section{Cel zadania}\indent\indent
		Naszym zadaniem było przewidywanie kursu walutowego EUR/USD (euro / dolar) na rynku Forex. W założeniach naszego projektu mieliśmy wykonać je na sieciach neuronowych.\\\indent
		Idea przewidywania kursu walutowego polega na analizie danych historycznych kursu oraz trendu, jakim wykres kursu zmierza. Celem naszego programu jest takie wytrenowanie sieci neuronowej, aby sama mogła przewidzieć przyszły kursy. Wartości te są niezwykle zmienne w czasie i trudne do przewidzenia, a w profesjonalnym oprogramowaniu, gdy  prawdopodobieństwo prawidłowego oszacowania przyszłego kursu przekracza 60\% , jest to uznawane za sukces. Korzyścią takiego programu jest oczywiście ułatwienie podejmowania decyzji finansowych.
		\section{Analiza problemu}\label{sec:analiza}\indent\indent
			Wartości kursów na rynku Forex zmieniają się z dokładnością co do sekund (tzw. \emph{tick}, który nie ma stałej wartości). By móc uprościć nasze zadanie zdecydowaliśmy się pobierać dane z dokładnością co do \textit{dnia}. W tej jednostce można wyróżnić 4 wartości kursu:
			\begin{itemize}
				\item \textbf{Open} - wartość otwarcia danego dnia
				\item \textbf{Low} - minimalna wartość w ciągu dnia
				\item \textbf{High} - maksymalna wartość w ciągu dnia
				\item \textbf{Close} - wartość zamknięcia danego dnia
			\end{itemize}
			Te cztery wartości z każdego dnia stanowią nasze dane wejściowe. Nie są to również jedyne parametry, które określają wartość kursu. Zdecydowaliśmy się naszej sieci przekazywać również współczynnik \textbf{RSI}\,(\emph{Relative Strength Index}), który określa siłę trendu. Oblicza się go w następujący sposób:
			$$ RSI=1-\cfrac{1}{1+\cfrac{A_k}{B_k}}$$
			gdzie:
			\begin{itemize}
				\item $ A_k $ - średnia wartość wzrostu cen zamknięcia z \textit{k} dni
				\item $ B_k $ - średnia wartość spadku cen zamknięcia z \textit{k} dni
			\end{itemize}
		\section{Wynik obliczeń}\indent\indent
			Nasza sieć decyduje czy wartość kursu zamknięcia następnego dnia będzie większa lub mniejsza od kursu zamknięcia dnia dzisiejszego. Zwrócona przez sieć wartość interpretowana jest następująco:
			\begin{itemize}
				\item $ 0 $ - wartość kursu będzie mniejsza
				\item $ 1 $ - wartość kursu będzie większa
			\end{itemize}
		\newpage
	\part{Specyfikacja zewnętrzna}
		\section{Interfejs graficzny}
			\begin{figure}[h!]
				\centering
				\includegraphics[width=0.90\textwidth]{./img/GUI}
				\caption{Interfejs po uruchomieniu programu}
			\end{figure}
			Przed przystąpieniem do trenowania sieci można są skonfigurować:
			\subsection{Struktura sieci}
				\begin{figure}[h!]
					\centering
					\includegraphics[width=0.40\textwidth]{./img/GUI_network_structure}
					\caption{Panel do konfiguracji struktury sieci z domyślnymi wartościami}
				\end{figure}
				W tym panelu definiuje się budowę wewnętrzną sieci neuronowej - liczbę warstw pośrednich oraz liczbę neuronów w każdej z warstw. Maksymalnie można utworzyć 3 warstwy z minimum 1 neuronem. W zależności od wartości w polu \emph{Hidden Layer Count}, odpowiednie pola \emph{Layer Neuron Count} są dostępne lub nie.
			\newpage
			\subsection{Zdolność nauki}
				\begin{figure}[h!]
					\centering
					\includegraphics[width=0.40\textwidth]{./img/GUI_initial_learning}
					\caption{Panel do konfiguracji zdolności nauki z domyślnymi wartościami}
				\end{figure}\indent
				Ten panel znajduje się w górnym lewym rogu interfejsu. Umożliwia konfigurację trzech parametrów:
				\begin{itemize}
					\item \emph{Initial Learning Rate} - wartość startowa zdolności nauki, z nią uruchamiany jest trening sieci.
					\item \emph{Final Learning Rate} - wartość zdolności nauki, którą sieć będzie miała, gdy trening się skończy.
					\item \emph{Learning Function} - specjalizowana funkcja, jakiej sieć backpropagacji będzie wykorzystywać do nauki. Są cztery możliwości:
					\begin{itemize}
						\item \emph{None} - standardowa funkcja
						\item \emph{Expotential} - funkcja eksponencjalna
						\item \emph{Hyperbolic} - funkcja hiperboliczna
						\item \emph{Linear} - funkcja liniowa
					\end{itemize}
				\end{itemize}
			\subsection{Inicjalizacja funkcji}
				\begin{figure}[h!]
					\centering
					\includegraphics[width=0.40\textwidth]{./img/GUI_network_initialization}
					\caption{Panel do konfiguracji inicjalizacji sieci z przykładowymi ustawieniami}
				\end{figure}
				W tej części (lewa strona interfejsu) można ustawić inicjalizację sieci, czyli początkową wartość wag neuronów we wszystkich warstwach. Istnieje cześć możliwości:
				\begin{itemize}
					\item \emph{None} - wartości domyślne.
					\item \emph{Constant} - wskazana wartość stała.
					\item \emph{NgyuenWidrow} - sparametryzowana funkcja\textit{ NGuyen Widrow}.
					\item \emph{NormalizedRandom} - znormalizowana wartość losowa.
					\item \emph{Random} - wartość losowa ze wskazanego zakresu.
					\item \emph{Zero} - funkcja typu Zero (wszystkie wartości są zbliżone do zera).
				\end{itemize}
			\newpage
			\subsection{Analiza plików}
				\begin{figure}[h!]
					\centering
					\includegraphics[width=0.40\textwidth]{./img/GUI_files}
					\caption{Panel z wczytanym plikiem i sposobem podziału}
				\end{figure}
				Ten panel znajduje się w górnej środkowej części interfejsu. Umożliwia pogląd załadowanego pliku do treningu sieci (tutaj \textit{EURUSD\_2015.csv}) oraz sposób w jaki dane będą grupowane:
				\begin{itemize}
					\item \emph{Monthly} - co miesiąc.
					\item \emph{Weekly} - co tydzień.
					\item \emph{Choose} - wg wskazanej przez użytkownika liczby dni.
				\end{itemize}
			\subsection{Trening sieci}
				\begin{figure}[h!]
					\centering
					\includegraphics[width=0.40\textwidth]{./img/GUI_network_training}
					\caption{Przykładowy wynik trenowania sieci po 100 iteracjach}
				\end{figure}
				Ten panel znajduje się w dolnym prawym rogu. Wymaga od użytkownika wprowadzania liczby iteracji treningowych, jakie wykona sieć (domyślnie jest to wartość 100). Po wciśnięciu przycisku \emph{Train} sieć rozpocznie swój trening. Gdy zostanie on zakończony, na dole pojawia się \emph{Mean Sqaured Error} (średni kwadratowy błąd) sieci w procentach. Oznacza on jakie wg obliczeń jest prawdopodobieństwo błędu sieci, czyli złego przewidzenia linii trendu kursu waluty.\\\indent
				Program poświęca 70\% próbek (zaokrąglonych w górę) do testów.
			\newpage
			\subsection{Trening sieci}
				\begin{figure}[h!]
					\centering
					\includegraphics[width=0.30\textwidth]{./img/GUI_testing}
					\caption{Przykładowy wynik testowania sieci}
				\end{figure}
				Na środku interfejsu znajduje się przycisk \emph{Predict away!}. Gdy sieć została wytrenowana i skonfigurowana, jego wciśnięcie powoduje uruchomienie testowania sieci i zmierzenie jej efektywności. Po wykonaniu testów informacje zostaną wyświetlone w tym samym panelu.\\\indent Program poświęca 30\% próbek w celu zbadania jakości działania sieci.
			\subsection{Graf błędu}
				\begin{figure}[h!]
					\centering
					\includegraphics[width=0.85\textwidth]{./img/GUI_graph}
					\caption{Przykładowy przebieg nauki sieci}
				\end{figure}
				W czasie treningu sieci jest generowany graf zależności jakości sieci od liczby iteracji. Przedstawia on jak prawdopodobieństwo błędu zmieniało się wraz z kolejnymi iteracjami. Można powiedzieć, że obrazuje on, w jaki sposób sieć się uczy.\\\indent
				Maksymalną wartością na osi poziomej jest liczba iteracji, z kolei na osi pionowej jest to maksymalny średni błąd kwadratowy (wyliczany dynamicznie w trakcie działania programu).
		\section{Korzystanie z programu}
			Aby móc wytrenować sieć i badać jej efekty, wystarczy wykonać kilka prostych kroków:
			\begin{enumerate}
				\item Wczytać plik z danymi próbkowymi. Należy wybrać z górnego menu \textit{File}$ \rightarrow $\emph{Open} i wskazać odpowiedni plik CSV.
				\item Ustawić wszystkie parametry zgodnie z instrukcjami w poprzednim podrozdziale.
				\item Wybrać \emph{Train} by wytrenować sieć.
				\item Wybrać \emph{Predict away!} by testować sieć.
			\end{enumerate}
	\part{Wykonanie i specyfikacja wewnętrzna}
		\section{Dane testowe}
			\subsection{Zbieranie}\indent\indent
				By móc przystąpić do działania, potrzebowaliśmy danych historycznych o kursie EUR/USD na rynku Forex. Istnieje dużo stron, które udostępniają te dane za odpowiednią opłatą, ale nie chcąc ani wydawać dziesiątek dolarów za nie, ani zakładać własnej firmy, szukaliśmy źródła, które udostępniłoby dane za darmo i przyjaznym formacie. Ostatecznie znaleźliśmy dwie strony internetowe, które nam pomogły:
				\begin{itemize}
					\item \url{http://www.global-view.com/forex-trading-tools/forex-history/} - strona za darmo udostępnia dane historyczne na temat kursu EUR/USD w formacie CSV. W generowanych przez nią plikach, z dokładnością do dnia, znajdują się kursy zamknięcia, najwyższy oraz najniższy w ciągu dnia. Prawie idealnie spełniało to nasze potrzeby.
					\item \url{http://pl.investing.com/currencies/eur-usd-historical-data} - strona za darmo wyświetla dane historyczne w postaci tabeli, której dane można kopiować. Mając dane z poprzedniej strony, wszystkie potrzebne nam wartości dziennych kursów otwarcia kopiowaliśmy ręcznie do pliku CSV.
				\end{itemize}
			\subsection{Format pliku}\label{subsec:format}\indent\indent
				Postanowiliśmy, żeby format pliku wejściowego był jak najprostszy, by dane mogły być szybko przetworzone, a jednocześnie wygodnie uzupełniane w miarę rozrastania się pliku. Zachowaliśmy plik CSV, jaki dostarczyła jedna ze stron, jest on wygodny oraz łatwy do sparsowania.\\
				Przykładowy kawałek pliku:
				\begin{lstlisting}
					2014-01-01,1.3745,1.3754,1.3744,1.38004
					2014-01-02,1.3655,1.3775,1.3627,1.37629
					2014-01-03,1.3587,1.3672,1.3583,1.36664
					2014-01-06,1.3634,1.3653,1.3569,1.35908
					2014-01-07,1.3614,1.366,1.3594,1.36289
					2014-01-08,1.3581,1.3635,1.3551,1.36153
				\end{lstlisting}
				Data stanowi unikalny klucz dziennego kursu i nie może się powtarzać. W tym rodzaju pliku nagłówki są zbędne, więc aby móc łatwo zapamiętać, która wartość jest która, ułożyliśmy je \textit{alfabetycznie}. Plik kolejno przedstawia:
				\begin{center}
					Data, \textbf{C}lose (zamknięcie), \textbf{H}igh (maksymalny), \textbf{L}ow (minimalny), \textbf{O}pen (otwarcie)
				\end{center}
			\subsection{Parsowanie pliku}\indent\indent
				Ponieważ istnieje spora liczba gotowych bibliotek do parsowania plików CSV, zdecydowaliśmy się nie pisać własnej funkcji, tylko skorzystać z jednej z już istniejących. Nie szukaliśmy długo, nasz wybór padł na małą, ale bardzo wysoko ocenianą bibliotekę "CSV Parser" autorstwa Ideafixxxer (prawdziwe imię i nazwisko nieznane) ze strony Code Project\footnote{Parser CSV: \url{http://www.codeproject.com/Tips/741941/CSV-Parser-Csharp}}.\\\indent
				Biblioteka przyjmuje na wejście plik CSV oraz zwraca strukturę typu \textbf{\color{blue}{string}[][]} z informacjami z pliku CSV w postaci tekstu. Format ten nie jest dogodny dla naszych potrzeb, dlatego napisaliśmy własny kod, który zmienia ową tablicę dwuwymiarową na mapę: kluczem jest data, a wartością tablica jednoelementowa wartości typu \textbf{\color{blue}{double}}. Całość została zakapsułkowana do klasy \emph{ExchangeData}. Co prawda po formacie próbek wyraźnie widać, że wystarczyłby typ danych \textbf{\color{blue}{float}}, jednak nasza biblioteka przyjmuje jako parametry wartości \textbf{\color{blue}{double}}, więc postanowiliśmy poświęcić tych kilka kilobajtów by uniknąć konieczności wielokrotnego rzutowania.
				\begin{lstlisting}
					class ExchangeData
					{
						private const int RATE_NUMBER = 4;
						private readonly Dictionary<DateTime, Double[]> data;
						
						public ExchangeData(string[][] inputData)
						{
							data = new Dictionary<DateTime, Double[]>();
							for (int i = 0; i < inputData.GetLength(0); i++)
							{
								DateTime dt = ToDateTime(inputData[i][0]);
								Double[] rate = new Double[RATE_NUMBER];
								// Close
								rate[0] = Double.Parse(inputData[i][1], CultureInfo.InvariantCulture);
								// High
								rate[1] = Double.Parse(inputData[i][2], CultureInfo.InvariantCulture);
								// Low
								rate[2] = Double.Parse(inputData[i][3], CultureInfo.InvariantCulture);
								// Open
								rate[3] = Double.Parse(inputData[i][4], CultureInfo.InvariantCulture);
								this.data[dt] = rate;
							}
						}
						
						public static DateTime ToDateTime(string datetime)
						{
							string year = datetime.Substring(0, 4);
							string month = datetime.Substring(5, 2);
							string day = datetime.Substring(8, 2);
							return new DateTime(Convert.ToInt32(year), Convert.ToInt32(month), Convert.ToInt32(day));
						}
					};
				\end{lstlisting}
				\vfill
				\theendnotes
	\newpage
	\section{Biblioteka NeuronDotNet}\label{sec:bib}
		\subsection{Szukanie rozwiązania}\indent\indent
			Decydując się na znany nam język programowania C\# potrzebowaliśmy biblioteki do wykonania programu. Kierując się dostępnymi źródłami w internecie, wybraliśmy bibliotekę NeuronDotNet. Jest to małe oprogramowanie Open Source napisane przez Brytyjczyka Vijeth D\footnote{Prawdziwe imię i nazwisko autora nie jest znane.}. Biblioteka jest mało popularna, ale bardzo wysoko oceniana. Jej rozwój został zatrzymany kilka lat temu, lecz do dla naszego zadania okazała się w zupełności wystarczająca.
		\subsection{Utworzenie sieci}\label{subsec:createNet}\indent\indent
			Z biblioteki NeuronDotNet wykorzystaliśmy sieć propagacji wstecznej (ang. \emph{backpropagation network}). Przedstawia ją klasa \textit{BackpropagationNetwork}. Aby utworzyć poprawnie sieć należy wykonać następujące kroki:
			\begin{enumerate}
				\item Utworzyć warstwę wejściową (np. typu \textit{Linear}), z parametrem \textit{integer} - liczbę neuronów warstwy.
				\begin{lstlisting}
					LinearLayer inputLayer = new LinearLayer(INPUT_NUMBER);
				\end{lstlisting}
				\item Utworzyć skończoną liczbę warstw pośrednich. Każdej warstwie pośredniej należy również przekazać liczbę neuronów jakie posiada w postaci liczby całkowitej. Dla usprawnienia kontroli, warstwy można trzymać na w liście (jak w poniższym przykładzie), ale nie jest to wymagane.
				\begin{lstlisting}
					List<SigmoidLayer> hiddenLayerList = new List<SigmoidLayer>();
					for (int i = 0; i < hiddenLayerCount; i++)
						hiddenLayerList.Add(new SigmoidLayer(neuronCountList[i]));
				\end{lstlisting}
				\item Utworzyć warstwę wyjściową.
				\begin{lstlisting}
					SigmoidLayer outputLayer = new SigmoidLayer(OUTPUT_NUMBER);
				\end{lstlisting}
				\item Mając utworzone wszystkie warstwy należy je połączyć obiektami klas typu \textit{Connector}. Nowe obiekty można po prostu wywołać instrukcją \textit{new}, bez konieczności przypisywania ich do czegokolwiek. \textit{Connector} przyjmuje dwa argumenty - warstwy, które ma połączyć. Na przykładzie warstwa wejściowa i pierwsza warstwa pośrednia.
				\begin{lstlisting}
					new BackpropagationConnector(inputLayer, hiddenLayerList[0]);
				\end{lstlisting}
				Tworząc połączenia można również wykorzystać ich inicjalizatory (\emph{initializers}), które ustawiają początkowe wagi synaps propagacji wstecznej w danym connectorze.
				\begin{lstlisting}
					new BackpropagationConnector(inputLayer, hiddenLayerList[0]).Initializer = new NormalizedRandomFunction();
				\end{lstlisting}
				Dostępnych jest 5 różnych typów inicjalizatorów:
				\begin{itemize}
					\item \emph{ConstantFunction}, przyjmuje jeden parametr typu \textit{double}
					\item \emph{NguyenWidrowFunction}, przyjmuje jeden parametr typu \textit{double}
					\item \emph{NormalizedRandomFunction}, bezparametrowa
					\item \emph{RandomFunction}, przyjmuje dwa parametry typu \textit{double}
					\item \emph{ZeroFunction}, bezparametrowa
				\end{itemize}
				Z poziomu GUI użytkownik dokonuje wyboru jednego z nich lub braku inicjalizatora połączenia.
				\vfill
				\theendnotes
				\newpage
				\item W powyższy sposób należy połączyć pozostałe utworzone warstwy w odpowiedniej kolejności, z inicjalizatorem lub bez.
				\begin{lstlisting}
					for (int i = 1; i < hiddenLayerCount; i++)
					{
						new BackpropagationConnector(hiddenLayerList[i - 1], hiddenLayerList[i]);
					}
					new BackpropagationConnector(hiddenLayerList[hiddenLayerCount - 1], outputLayer);
				\end{lstlisting}
				\item W ostatnim kroku należy utworzyć naszą sieć jako nowy obiekt klasy \textit{BackpropagationNetwork}. Konstruktor przyjmuje dwa parametry - warstwę wejściową i wyjściową. Ponieważ warstwy pośrednie i połączenia między nimi zostały zdefiniowane wcześniej, obiekt \textit{ourNetwork} jest w pełni gotowy do pracy.
				\begin{lstlisting}
					BackpropagationNetwork ourNetwork = new BackpropagationNetwork(inputLayer, outputLayer);
				\end{lstlisting}
				\item Tak przygotowaną sieć można wytrenować przekazując jej przygotowany zestaw treningowy (szczegóły w Trenowanie sieci \ref{subsec:train}), a następnie wykorzystać.
			\end{enumerate}
		\subsection{Zdolność nauki sieci}\indent\indent
			Po utworzeniu sieci, biblioteka NeuronDotNet umożliwia zmianę kilku parametrów, które wpływają na efekty treningu. Są to wartości związane ze zdolnościami nauki sieci, czyli określenie specjalizowanej funkcji do nauki oraz jej dwóch parametrów: zdolności startowej nauki, z którą uruchamiany jest trening sieci oraz końcowej, którą sieć będzie mieć pod koniec treningu, oba typu \textit{double}.\\\indent
			Funkcja może być standardowa, wywołana w ten sposób:
			\begin{lstlisting}
				BackpropagationNetwork ourNetwork;
				ourNetwork.SetLearningRate(initialLearningRate, finalLearningRate);
			\end{lstlisting}
			Można również wybrać jedną z dostępnych funkcji:
			\begin{itemize}
				\item \emph{ExpotentialFunction} - funkcja eksponencjalna
				\item \emph{HyperbolicFunction} - funkcja hiperboliczna
				\item \emph{LinearFunction} - funkcja liniowa
			\end{itemize}
			\begin{lstlisting}
				BackpropagationNetwork ourNetwork;
				ourNetwork.SetLearningRate(new ExponentialFunction(initialLearningRate, finalLearningRate));
			\end{lstlisting}
			Wszystkie trzy parametry są możliwe do ustalenia przed użytkownika w interfejsie graficznym.
	\section{Algorytmy działania programu}
	\subsection{Wstępne przetworzenie danych}\label{subsec:przetworzenie}\indent\indent
		Nasze dane wejściowe są dzielone na dwie części: 70\% jest poświęconych dla trenowania sieci, z kolei pozostałe 30\% dla testowania. W obu grupach dane próbkowe są wyliczane tak samo. Użytkownik podaje przedział czasowy wg którego chce badać dane, muszą to być co najmniej 3 dni (okresy) oraz mniej niż całkowita liczba danych wejściowych. W momencie przystąpienia do treningu i testowania, dane są wstępnie przetwarzane dla złagodzenia linii trendu i efektywniejszych obliczeń.\\\indent
		Dla \textit{n} danych wejściowych i podziale równym \textit{k} okresów, gdzie $ n > k $, dane są dzielone na $ (n-k) $ grup. W obrębie \textit{k}-licznej grupy wyliczana jest średnia krocząca wykładnicza z $ k-1 $ argumentów, gdzie ten z najbliższą datą jest najważniejszy. Argument \textit{k}-ty z kolei stanowi wartość oczekiwaną. Procedura ta jest wykonywana dla każdej z czterech dziennych wartości kursu.\\\indent
		Przykładowo, posiadając 200 próbek dzielonych po 10 okresów, tworzymy 190 grup danych. W każdej z grup liczymy średnią wykładniczą z 9 argumentów, a ostatni jest wartością oczekiwaną. 70\% tych grup będzie stanowiło zestaw treningowy dla sieci.\\\indent
		Wektor wejściowy każdej próbki treningowej składa się z pięciu parametrów:
		\begin{itemize}
			\item Cztery z nich to różnice pomiędzy aktualnymi kursami zamknięcia, najniższym, najwyższym oraz otwarcia, a między średnimi wykładniczymi tych kursów z danego okresu. Wartości średnie są wyliczane przed wywołaniem funkcji \textit{Train} dla każdej z próbek.
			\item Piątym parametrem jest współczynnik RSI obliczony dla tego samego okresu.
		\end{itemize}\indent\indent
		Z kolei wektor wyjściowy to różnica pomiędzy jutrzejszym kursem zamknięcia, a aktualnym kursem zamknięcia. Jeśli jest on większy od zera oznacza wzrost, a jeśli mniejszy to spadek wartości kursu.
	\subsection{Trenowanie sieci}\label{subsec:train}
	\begin{lstlisting}
	public void Train()
	{
		// Utworzenie sieci neuronowej
		CreateNetwork();
		// ustawienie mocy poznawczej
		SetNetworkLearningRate();
		TrainingSet trainingSet = new TrainingSet(INPUT_NUMBER, OUTPUT_NUMBER);
		// 70% danych do treningu
		trainingDataCount = (int)Math.Ceiling(inputData.Count * 0.7);
		for (int i = 0; i < trainingDataCount; ++i)
		{
			trainingSet.Add(
				new TrainingSample(
					// wektor wej. - różnice między aktualną wartością, a średnią
					new double[INPUT_NUMBER]
					{
						actualClose[i] - meanClose[i], actualHigh[i] - meanHigh[i], actualLow[i] - meanLow[i], actualOpen[i] - meanOpen[i], RSI[i]
					},
					// wektor wyj. - różnica między jutrzejszym kursem zamknięcia, a aktualnym
					new double[1]
					{
						tomorrowClose[i] - actualClose[i]
					}
				)
			);
		}
		// nauka sieci
		ourNetwork.Learn(trainingSet, iterations);
		ourNetwork.StopLearning();
	}
	\end{lstlisting}
	Funkcja \textit{Train} jest wywoływana po naciśnięciu przycisku Train w interfejsie użytkownika.  Najpierw tworzy nową sieć neuronową utworzoną zgodnie z biblioteką (patrz: tworzenie sieci w NeuronDotNet \ref{subsec:createNet}) oraz inicjuje ją wartościami z wybranymi przez użytkownika.\\
	Następnie funkcja tworzy dane testowe, na które poświęca 70\% próbek danych (zaokrąglone w górę). Zestaw treningowy (\emph{training set}) musi mieć parametry zgodne z siecią - liczba wejść jest równa liczbie neuronów w warstwie wejściowej, a liczby wyjść liczbie neuronów z wyjściowej.\\
	Do zestawu treningowego wprowadzane są próbki treningowe (\emph{training sample}). Ich wektory wejściowe i wyjściowe są tworzone zgodnie z algorytmem opisanym we wstępnym przetworzeniu danych\ref{subsec:przetworzenie}. Gdy wszystkie próbki zostaną utworzone rozpoczyna się nauka sieci oraz przekazanie jej zestawu treningowego oraz liczby iteracji sieci.
	\subsection{Obliczanie współczynnika RSI}
	\begin{lstlisting}
	private double calculateRSI(List<Double[]> input, int start = 0)
	{
		int periodNumber = input.Count - start;
		double alpha = 2.0d / (periodNumber + 1);
		double denominator = 0.0d;
		double avg_growth = 0.0d, avg_fall = 0.0d;
		List<double> growth = new List<double>();
		List<double> fall = new List<double>();
		// porównanie wartości Close pomiędzy kolejnymi dniami oraz obliczenie spadku/wzrostu
		for (int i = start; i < periodNumber - 1; ++i)
		{
			double weight = Math.Pow(1 - alpha, periodNumber - i);
			double value = input[i][0];
			double value_next = input[i + 1][0];
			if ((value - value_next) >= 0)
				fall.Add((value - value_next) * weight);
			else
				fall.Add(0);
			if ((value_next - value) >= 0)
				growth.Add((value_next - value) * weight);
			else
				growth.Add(0);
			denominator += weight;
		}
		//obliczanie sredniego spadku/wzrostu
		avg_fall = fall.Sum() / denominator;
		avg_growth = growth.Sum() / denominator;
		//oblicznie RSI
		double RSI = 100 - (100 / (1 + (avg_growth / avg_fall)));
		return RSI;
	}
	\end{lstlisting}
	Metoda zwracająca współczynnik RSI (patrz: Analiza Problemu \ref{sec:analiza}). W celu obliczenia średniego wzrostu i spadku kursu wykorzystano pętle, która iteruje po podanej liście zawierającej dane historyczne kursów (parametr \emph{input}). Zależnie od tego czy różnica między wartością Close aktualnego dnia, a dnia następnego jest dodatnia czy ujemna, następuje dodanie tej różnicy do odpowiedniej listy (spadków / wzrostów). W przypadku wartości spadku różnica obliczana jest między dniem następnym, a aktualnym, po to aby wartości w obu listach były dodatnie.\\
	W ciele funkcji odnosimy się do elementu zerowego tablicy, ponieważ właśnie tam znajdują się kursy zamknięcia (patrz: format pliku \ref{subsec:format}).\\
	Po wyjściu z pętli następuje obliczenie średnich z wartości znajdujących się w listach, a następnie podstawienie tych wartości do wzoru (patrz: Analiza Problemu \ref{sec:analiza}) i zwrócenie jej przez metodę.
	\newpage
	\subsection{Testowanie sieci}
		\begin{lstlisting}
			public int Run(Int32 diff)
			{
				//tablica wyników przewidywan
				double[] result = new double[inputData.Count - trainingDataCount];
				//true jesli wynik przewidywania i aktualne dane sie pokrywaja
				bool[] correctPredict = new bool[inputData.Count - trainingDataCount];
				int j = 0;
				//30% danych do testowania
				for (int i = trainingDataCount; i < inputData.Count - diff; ++i)
				{
					result[j] = ourNetwork.Run(
					new double[INPUT_NUMBER]
					{
						actualClose[i] - meanClose[i], actualHigh[i] - meanHigh[i], actualLow[i] - meanLow[i], actualOpen[i] - meanOpen[i], RSI[i]
					})[0];
					if ((tomorrowClose[i] - actualClose[i]) > 0 && result[j] > 0)
						correctPredict[j] = true;
					else if ((tomorrowClose[i] - actualClose[i]) <= 0 && result[j] <= 0)
						correctPredict[j] = true;
					else
						correctPredict[j] = false;
					j++;
				}
				int true_count = 0;
				for (int i = 0; i < correctPredict.Length; ++i)
					if (correctPredict[i])
						true_count++;
				return true_count;
			}
		\end{lstlisting}
		Metoda przekazująca 30\% danych wejściowych do sieci, zwracająca liczbę poprawnie przewidzianych dni. Składa się głownie z pętli, w której przekazuje się kolejne zestawy danych do sieci neuronowej, a następnie zapisuje zwracany wynik do listy (\textit{result}). Zwrócony wynik zostaje porównany z rzeczywistą wartością z zestawu danych - sprawdzane jest czy przewidziany kurs jest prawidłowo większy lub mniejszy od poprzedniego. Jeśli równanie jest spełnione następuję zapisanie do list wartości \textit{true}, jeśli nie to \textit{false}. Po wyjściu z pętli następuje podliczenie prawidłowych prognoz  w zmiennej (\textit{true\_count}), która jest wartością zwracaną przez metodę po wyjściu z pętli. Wynik ten jest liczbą poprawnie przewidzianych kursów.\\\\
		Testując sieć neuronową przesyła się na jej warstwę wejściową różnice między aktualnymi wartościami kursów, a ich średnimi z danego zakresu (wybór użytkownika), a także wartość RSI z danego zakresu.
		\part{Testy i wnioski}
		
\end{document}